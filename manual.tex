\documentclass[11pt,a4paper]{article}

\usepackage[T1]{fontenc}
\usepackage[latin1]{inputenc}
\usepackage{fullpage}
\usepackage{url}

\newcommand{\WEB}{\textsf{WEB}}
\newcommand{\Caml}{\textsf{Caml}}
\newcommand{\OCaml}{\textsf{Objective Caml}}
\newcommand{\ocamlweb}{\textsf{ocamlweb}}
\newcommand{\monurl}[2]{\url{#1}}
%HEVEA\renewcommand{\monurl}[2]{\url{#2}{#2}}

\begin{document}

\title{BibTeX2HTML: a translator of BibTeX bibliographies into HTML}
\author{Jean-Christophe Filli\^{a}tre and Claude March\'e \\
        \normalsize\monurl{http://www.lri.fr/~filliatr/bibtex2html}{http://www.lri.fr/\~{}filliatr/bibtex2html}}
\date{}
\maketitle

\tableofcontents


\section{Introduction}


bibtex2html is a BibTeX to HTML translator. It is invocated as 
\begin{quote}
\texttt{bibtex2html} [options] \textit{file.bib}
\end{quote}
where the possible 
%HEVEA <a href="#options">
options
%HEVEA </a>
are described below and where
\textit{file.bib} is the name of the BibTeX file, which must have a
\textit{.bib} suffix.

Then two HTML documents are created: 
\begin{itemize}
\item \textit{file.html} which is the bibliography in HTML format
%HEVEA, like <a href=examples/biblio-these.html>this</a>
;
\item \textit{file-bib.html} which contains all the entries in ASCII
  format.
%HEVEA ,like <a href=examples/biblio-these-bib.html>this</a>
\end{itemize}
\texttt{bibtex} is called on \textit{file.bib} in order to produce the
a LaTeX document, and then this LaTeX document is translated into an
HTML document.  The BibTeX file \textit{file.bib} is also parsed in
order to collect additional fields (abstract, url, ps, http, etc.)
that will be used in the translation.

%HEVEA <a name=fields>
\subsection{Additional fields and automatic web links}

The main interest of \texttt{bibtex2html} with respect to a
traditional LaTeX to HTML translator is the use of additional fields
in the BibTeX database and the automatic insertion of web
links.

A link is inserted:
\begin{itemize}

\item  at each cross-reference inside the bibliography entries; 

\item  when the \verb|\url| LaTeX macro is used in the text; 

\item  for each BibTeX field whose name is "ftp", "http", "url", "ps"
  ,"dvi", "rtf", "pdf",
  "documenturl", "urlps" or "urldvi". The name of this link depends on
  the nature of the link or file specified: 
  \begin{itemize}
  \item a suffix \texttt{.dvi} will give a link named "DVI";
  \item a suffix \texttt{.ps} will give a link named "PS";
  \item a suffix \texttt{.pdf} will give a link named "PDF";
  \item a suffix \texttt{.rtf} will give a link named "RTF";
  \item any other filename will give a link named "Available here".
  \end{itemize}
  
  An additional suffix \texttt{.gz}, \texttt{.Z} or
  \texttt{.zip} will add the prefix "Compressed" before the
  file format.

  The file format is prefixed 
  You can insert web link for other fields with the option
  \texttt{-f} (see below).
  
\end{itemize}


%HEVEA See the result on <a href=examples/biblio-these.html>this example</a>.

\subsubsection{Abstracts}

If a BibTeX entry contains a field \texttt{abstract} then its contents
is quoted right after the bibliography entry.
%HEVEA like <a href=examples/publis-abstracts.html>this</a>. 

This behavior may be suppressed with the option \texttt{--no-abstract}.

If you want both versions with and without abstracts, use the option
\texttt{--both}. In that case, links named "Abstract" will be
inserted from the page without abstracts to the page with abstracts,
%HEVEA like <a href=examples/publis.html>this</a>.

%HEVEA <a name=options>
\subsection{Command line options}

Most of the command line options have a short version of one character
(e.g. \texttt{-r}) and an easy-to-remember/understand long version 
(e.g. \texttt{--reverse-sort}).

\subsubsection{General aspect of the HTML document}

\begin{description}
  
\item[\texttt{-t} \textit{string}, \texttt{--title} \textit{string}]
  specify the title of the HTML file (default is the file name).

  
\item[\texttt{-s} \textit{string}, \texttt{--style} \textit{string}]
  use BibTeX style \textit{string} (plain, alpha, etc.).  Default
  style is plain.

  
\item[\texttt{-noabstract}, \texttt{--no-abstract}] do not print the
  abstracts (if any).

  
\item[\texttt{-both}, \texttt{--both}] produce both pages with and
  without abstracts. If the BibTeX file is foo.bib then the two pages
  will be respectively foo.html and foo-abstracts.html (The suffix may
  be different, see option \texttt{--suffix}). Links are inserted from
  the page without abstracts to the page with abstracts.

  
\item[\texttt{-nokeys}, \texttt{--no-keys}] do not print the cite
  keys.

  
\item[\texttt{-f} \textit{field}, \texttt{--field} \textit{field}] add
  a web link for that BibTeX field.

  
\item[\texttt{-multiple}, \texttt{--multiple}] make a separate web
  page for each entry.  \textit{Beware: this option produces as many
    HTML files as BibTeX entries!}


\end{description}


\subsubsection{Controlling the translation}

\begin{description}
  
\item[\texttt{-m} \textit{file}, \texttt{--macros-from} \textit{file}]
  read the LaTeX macros in the given file.


\end{description}


\subsubsection{Sorting the entries}

\begin{description}
  
\item[\texttt{-d}, \texttt{--sort-by-date}] sort by date.

  
\item[\texttt{-a}, \texttt{--sort-as-bibtex}] sort as BibTeX (usually
  by author).

  
\item[\texttt{-u}, \texttt{--unsorted}] unsorted i.e. same order as in
  .bib file (default).

  
\item[\texttt{-r}, \texttt{--reverse-sort}] reverse the sort.

  
\item[\texttt{-e} \textit{key}, \texttt{--exclude} \textit{key}]
  exclude an entry.


\end{description}


\subsubsection{Miscellaneous options}

\begin{description}
  
\item[\texttt{-nodoc}, \texttt{--nodoc}] do not produce a full HTML
  document but only its body (useful to merge the HTML bibliography in
  a bigger HTML document).

  
\item[\texttt{-suffix} \textit{string}, \texttt{--suffix}
  \textit{string}] give an alternate suffix \textit{string} for the
  HTML documents (default is \texttt{.html}).

    
\item[\texttt{-c} \textit{file}, \texttt{--command} \textit{file}]
  specify the BibTeX command (default is \texttt{bibtex}).  Useful if
  \texttt{bibtex} is not in your path.

  
\item[\texttt{-i}, \texttt{--ignore-errors}] ignore BibTeX errors.

  
\item[\texttt{-h}, \texttt{--help}] print a short usage and exit.

  
\item[\texttt{-v}, \texttt{--version}] print the version and exit.


\end{description}


%HEVEA <hr>
%HEVEA <img src="http://www.lri.fr/~filliatr/icons/mail.gif" ALIGN=middle>
%HEVEA <em><a href="mailto:Jean-Christophe.Filliatre@lri.fr">
%HEVEA Jean-Christophe.Filliatre@lri.fr</a> , Fri Feb 12 17:46:23 1999 
%HEVEA .</em>

\end{document}

%%% Local Variables: 
%%% mode: latex
%%% TeX-master: t
%%% End: 
